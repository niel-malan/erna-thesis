% Appendix Template

\chapter{Thermocouple Production} % Main appendix title

\label{AppendixC} % Change X to a consecutive letter; for referencing this appendix elsewhere, use \ref{AppendixX}

How the thermocouple probe was made:

\begin{enumerate}
  \item 
	\item A piece of 0.25 mm polyimide-coated fused-silica capillary of about 500 mm length was cut and mounted with sticky tape on a wooden metre stick.
	\item A longer length of 0.1 mm fused-silica capillary was threaded through the 0.25 mm capillary.
	\item The end of one of the thermocouple wires was inserted into the end of the 0.1 mm capillary. A drop of cyanoacrylate adhesive was touched to the end of the capillary. Capillary action drew the liquid adhesive into the capillary and fixed the wire in place.
	\item The wire was drawn carefully into the capillary by pulling on the 0.1 mm capillary.
	\item Once the end of the wire protruded through the end of the 0.25 mm capillary the end of the 0.1 mm capillary was cut off.
	\item The wire was anchored at one end with adhesive tape, pulled tight, and anchored at the other end. 
	\item The procedure was repeated for the other wire.
	\item The two thermocouple wires (Goodfellow) was clamped in a twisting bar. The twister bar has a square profile, 8 mm on a side.
	\item The wires were flamed with a cigar lighter until they were red a dull red hot. (At any higher temperature the wires would melt.) This chars the polyimide coating.
	\item The flamed portion of the wires were lightly sanded with 1200 grit water paper. A pair of small pliers had its beak lined with the abrasive, and lightly stroked up and down the wire to remove the char.
	\item A 6 mm tube was inserted between the wires to serve as spacer. and moved until about 10mm away from the twister bar.
	\item The wires were twisted by turning the twister bar until the twister portion was about 5mm long.
	\item The spools were rewound to retract the wire, until the start of the twist rested on the clamping bar.
	\item The clamping weight was lowered onto the clamping bar, keeping the pair or wires in place.
	\item A small pair of scissors was used to snip off the end of the
	\item The welding electrode was brought into position. This was a carbon rod in the form of a pencil lead (Schwann Stabilo), 2 mm in diameter, mounted on a screwing connector. The welding circuit consisted of a bench power supply set to approximately 20V. connected. The negative terminal was clamped to the aluminium base plate of the microscope, on which rested the brass clamp bar. The positive terminal was clamped to the carbon electrode. A voltage of approximately 23 V was applied.
	\item It was discovered that the carbon electrode should not have a polished end, but a roughly broken end. 
	\item The electrode was moved closer to the clamped twisted wire.
	\item At the right point a spark would jump from the carbon to the wire, melting the end of the wire. The molten wire would draw into a globule on the end of the wire, withdrawing from the electrode and so breaking the spark, ending the heating.
	\item If the wire would actually touch the electrode the wire would heat up red hot and melt off, usually destroying the twist and requiring making a new twist.
	\item If all went well, there would be a hemispherical weld at the end of the twist where the two wires would be joined.
	\item The thermocouple wires was withdrawn into the capillary until the end just protruded, kept in place with a pair of rubber-tipped self-closing tweezers.
	\item If two sets of thermocouples were needed, the procedure would be repeated for another pair of wires.
	\item The wires would be pulled back, one pair at a time.
	\item The other end of the capillary was taped to the connector pad.
	\item The wires was flamed and scraped to remove the polyimide isolation, and screwed down on a screw connector block.
	\item The resistance between the protruding end of the thermocouple and the connector block was measured to ensure electrical connection. The resistance for the Chromel is 1440 $\Omega{m}^{-1}$, and for the Alumel 600 $\Omega{m}^{-1}$
	\item A drop of cyanoacrylate adhesive was put on the end of the capillary to anchor the wires and to prevent high-pressure gas from blowing the wires out of the capillary. 
\end{enumerate}
