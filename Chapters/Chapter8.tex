% Chapter 8

\begin{savequote}[45mm]
In a book that I took from a shelf \ldots
\qauthor{Ancient proverb}
\end{savequote}

\chapter{Conclusion} % Main chapter title

\zlabel{chap:pdfstartpage} % For hyperref: this makes this the default page to open when the PDF opens.
\todo{Move zlabel{chap:pdfstartpage} to appropriate chapter. Remove before publication}
\todo{Remove usepackage{stampinclude} before publication.}

\label{Chapter8} % For referencing this chapter elsewhere, use \ref{Chapter8}


%----------------------------------------------------------------------------------------
%	SECTION 1
%----------------------------------------------------------------------------------------

\section{Introduction}

It has been shown that an SFC×GC instrument can be built using mainly 20th century technology. 

\section{Synopsis}

\subsection{Chapter \ref{Chapter1}: \nameref{Chapter1} }

The molecular basis of sustainability \autocite{Anastas2016} demands that
chemists discuss sustainability in terms of compounds. Chapter \ref{Chapter1}
opens with the idea that energy is an essential component of industrialized
societies, but that choosing of fossil fuels as our source of energy causes
pollution, which threatens to nullify the benefits they bring. It describes the
process which causes carbon dioxide to be a major emittant, and how its
interaction with planetary radiation makes it a pollutant. The concept of
``carbon footprint'' is introduced, which allows the comparison of activities in
terms of their carbon pollution, which allows decision makers to select the
least polluting option. The discussion then focuses on internal-combustion
engines, which is a major source of carbon and noxious pollution, and it is
shown that higher-efficiency engines have lower carbon footprints. A discussion
of ways to reduce the carbon footprint of internal combustion engines shows
that, in the cases where they cannot be replaced by electric motors, a reduction
of carbon footprint can be obtained by preferring large, high-performance Diesel
engines fuelled by biodiesel. The discussion concludes with the idea that the
success of such engines will demand high quality biodiesel, and that
chromatography will plary a central role in ensuring that quality.

\subsection{Chapter \ref{Chapter2}: \nameref{Chapter2} }

Chapter \ref{Chapter2} starts with a discussion of the chemical industry and the
need to move towards ``green'' chemistry. It introduces carbon dioxide as a
renewable resource, discusses its various uses in industry, and then focuses on
its application in extraction and chromatography. It describes how carbon
dioxide becomes a solvent at high pressures and densities, and then introduces
supercritical fluid chromatography (SFC). Fractions of eluate from SFC can be
analysed by gas chromatography, and if a suitable set of criteria is met, then
the combination is called SFC×GC.

\subsection{Chapter \ref{Chapter3}: \nameref{Chapter3} }

As discussed in Chapter \ref{Chapter1}, a reliable high-performance engine
requires a reliable fuel. Chapter \ref{Chapter3} discusses the concepts of
technical standards, which establishes requirements with which fuels must
comply. The discussion then focuses on the technical standard SANS 1935, which
lists the requirements that biodiesel offered for sale in South Africa must
comply with, concluding with the 

\subsection{Chapter \ref{Chapter4}: \nameref{Chapter4} }

Chapter \ref{Chapter4} explains the experimental equipment used for
chromatography using high-pressure carbon dioxide as a mobile phase. It starts
with describing the mobile phase, how it is stored and pumped, how modifier is
added, and how the sample is injected. It describes problems with designing the
restrictor that maintains the pressure, and concludes with remarks about using a
gas chromatograph as a detector.

\subsection{Chapter \ref{Chapter5}: \nameref{Chapter5} }

Chapter \ref{Chapter5} opens with a discussion on the time aspect of
comprehensive SFC×GC chromatography, and shows that for practical analysis the
GC dimension must be \textit{fast}. The theory of fast GC is discussed, which
leads to the the need for fast temperature programming. The design of a
resistively heated coaxial heater is described, including its calibration and
control. The discussion then covers the need for a cold column, and the design
of coaxial cooling using boiling liquid carbon dioxide is described. Next, the
discussion covers the design of hardware to mount the coaxial heater in a
conventional GC oven. The chapter concludes with a description of the detector
and the data flow from signal to final chromatogram.

\subsection{Chapter \ref{Chapter6}: \nameref{Chapter6} }

Chapter \ref{Chapter6} discusses the application of the developed SFC×GC
instrument to study the fatty acid profile of various potential biodiesel
feedstocks.

\subsection{Chapter \ref{Chapter7}: \nameref{Chapter7} }

Chapter \ref{Chapter7} demonstrates that the use of SFC with modifiers does not
preclude the use of the flame ionization detector when GC is used as a second
dimension.

\section{Strengths of the SFC×GC}

\subsection{Length of SFC column}

We could separate paraffins, olefins and aromatics, using the long column.

\subsection{Variable modulation period}

It is possible to vary the modulation period during an SFC×GC run. This would
allow one to speed up analysis, by collecting fewer fractions where less
information would be expected.

\section{Limitations of this SFC×GC design}

\subsection{}

\section{Suggested design improvements}

\subsection{Four-wire resistance measurement}

The resistive heater had a design that depended on measuring the resistance of
the thin-walled stainless steel tube. The resistance was measured by comparing
the potential difference between the ends of the heater with the potential
difference over a reference resistor carrying an identical current. The circuit
that carried the current also measured the potential difference. Because the
current was high, the circuit had to be constructed in such a way that it no
other significant resistance in it, which meant that care had to be taken to
used heavy-guage cable and only use soldered joints. 

These troubles could have been avoided if the current circuit and the measuring
circuit were separated, using 'four wire' resistance measurement. In such a
design the potential difference is measure using a circuit that connects
directly to the voltmeter. The voltmeter has a high input impedance, which means
that the current that the circuit carries is low, and stray resistances will
play much less of a role. Also, this would have made it possible to use bolted
joints on the current-carrying circuit, which could have simplifed operations.

\subsection{Legs heating and cooling integrated in detector and inlet}


\subsection{Siltek-treated coaxial heater}


\todo{Discuss material that passes through column during trapping.}

\todos
