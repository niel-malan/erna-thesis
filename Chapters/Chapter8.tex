% Chapter 8

\begin{savequote}[45mm]
In a book that I took from a shelf \ldots
\qauthor{Ancient proverb}
\end{savequote}

\chapter{Conclusion} % Main chapter title

\label{Chapter8} % For referencing this chapter elsewhere, use \ref{Chapter8}

%----------------------------------------------------------------------------------------
%	SECTION 1
%----------------------------------------------------------------------------------------

\section{Introduction}

It has been shown that an SFC×GC instrument can be built using mainly 20th century technology. 

\section{Synopsis}

\subsection{Chapter \ref{Chapter1}: \nameref{Chapter1} }

Chapter \ref{Chapter1} offers a broad view of the context in which biodiesel
will find an application in a future with a sustainable economy.

\subsection{Chapter \ref{Chapter2}: \nameref{Chapter2} }

Chapter \ref{Chapter2} introduces carbon dioxide as a renewable resource, and 

\subsection{Chapter \ref{Chapter3}: \nameref{Chapter3} }

Chapter \ref{Chapter3} discusses the technical standards with which biodiesel must comply 

\subsection{Chapter \ref{Chapter4}: \nameref{Chapter4} }

Chapter \ref{Chapter4} explains the experimental equipment used for
chromatography using high-pressure carbon dioxide as a mobile phase.

\subsection{Chapter \ref{Chapter5}: \nameref{Chapter5} }

Chapter \ref{Chapter5} discusses the development of a fast
temperature-programmed gas chromatograph, based on a resistive heater coaxial to
the capillary column, that uses precisely-applied evaporating carbon dioxide as
powerful coolant to reduce time between GC runs to a minimum.

\subsection{Chapter \ref{Chapter6}: \nameref{Chapter6} }

Chapter \ref{Chapter6} discusses the application of the developed SFC×GC
instrument to study the fatty acid profile of various potential biodiesel
feedstocks.

\subsection{Chapter \ref{Chapter7}: \nameref{Chapter7} }

Chapter \ref{Chapter7} demonstrates that the use of SFC with modifiers does not
preclude the use of the flame ionization detector when GC is used as a second
dimension.

\section{Strengths of the SFC×GC}

\section{Limitations of this SFC×GC design}

\subsection{}

\section{Suggested design improvements}


\todo{Discuss material that passes through column during trapping.}

\subsection{Four-wire resistance measurement}
\subsection{Legs heating and cooling integrated in detector and inlet}
\subsection{Siltek-treated coaxial heater}

