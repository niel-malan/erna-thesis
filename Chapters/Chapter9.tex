% Chapter 9

\begin{savequote}[\quotewidth]
``Begin at the beginning," the King said, very gravely, ``and go on till you come to the end: then stop''.
\qauthor{Lewis Carroll}
\end{savequote}

\chapter{Conclusion} % Main chapter title

%\zlabel{chap:pdfstartpage} % For hyperref: this makes this the default page to open when the PDF opens.
%\todo{Move zlabel{chap:pdfstartpage} to appropriate chapter. Remove before publication}

In this thesis the research conducted on the practical implementation of SFC×GC
as a chromatographic technique is discussed. In this final chapter we provide an
overview of the project and discuss some of the lessons learned.

\section{Synopsis}

\subsection{Chapter \ref{Chapter1}: \nameref{Chapter1} }

The molecular basis of sustainability \autocite{Anastas2016} demands that
chemists discuss sustainability in terms of chemical compounds. Chapter
\ref{Chapter1} of this thesis opens with the idea that energy is an essential
component of industrialized societies, but that choosing fossil fuels as our
source of energy causes pollution, which threatens to nullify the benefits they
bring. It describes the process which causes carbon dioxide to be emitted in
large quantities, and how its interaction with planetary radiation makes it a
pollutant. The concept of a ``carbon footprint'' is introduced, which allows the
comparison of activities in terms of their carbon pollution, enabling decision
makers to select the least polluting option. The discussion then focuses on
internal-combustion engines, which is a major source of carbon and noxious
pollution, and shows that higher-efficiency engines have lower carbon
footprints. A discussion of ways to reduce the carbon footprint of internal
combustion engines shows that, in the cases where they cannot be replaced by
electric motors, a reduction of carbon footprint can be obtained by preferring
large, high-performance diesel engines fuelled by biodiesel. The discussion
concludes with the idea that the success of such engines will demand high
quality biodiesel, and that chromatography will play a central role in ensuring
that quality.

\subsection{Chapter \ref{Chapter2}: \nameref{Chapter2} }

Chapter \ref{Chapter2} starts with a discussion of the chemical industry and the
need to move towards ``green'' chemistry. It introduces carbon dioxide as a
renewable resource, discusses its various uses in industry, and then focuses on
its application in extraction and chromatography. It describes how carbon
dioxide becomes a solvent at high pressures and densities, and then introduces
supercritical fluid chromatography (SFC). Fractions of eluate from SFC can be
analysed by gas chromatography, and if a suitable set of criteria is met, then
the combination is called SFC×GC.

\subsection{Chapter \ref{Chapter3}: \nameref{Chapter3} }

As discussed in Chapter \ref{Chapter1}, a reliable high-performance engine
requires a reliable fuel. Chapter \ref{Chapter3} discusses the concepts of
technical standards, which establish requirements fuels must comply with to be
reliable. The discussion then focuses on the technical standard SANS 1935, which
lists the requirements that South African commercial biodiesel must comply with,
concluding with the role chromatography plays in the process of ensuring
compliance.

\subsection{Chapter \ref{Chapter4}: \nameref{Chapter4} }
 
Chapter \ref{Chapter4} explains the experimental equipment used for
chromatography using high-pressure carbon dioxide as a mobile phase. It starts
with describing the mobile phase, how it is stored and pumped, how modifier is
added, and how the sample is injected. It describes problems with designing the
restrictor that maintains the pressure, and concludes with remarks about using a
gas chromatograph as a detector for SFC. 

\subsection{Chapter \ref{Chapter5}: \nameref{Chapter5} }

Chapter \ref{Chapter5} opens with a discussion on the time aspect of
comprehensive SFC×GC chromatography, and shows that for practical analysis the
GC dimension must be \keyword{fast}. The theory of fast GC is discussed, which
leads to the need for fast temperature programming. The design of a resistively
heated coaxial heater for short GC columns is described, including its
calibration and control. The discussion then covers how a cold GC column acts as
a trap, and the design of coaxial cooling using boiling liquid carbon dioxide is
described. Next, the discussion covers the design of hardware to mount the
coaxial heater of the fast GC in a conventional GC oven. The chapter concludes
with a description of the flame ionization detector and the data flow from
signal to final chromatogram.

\subsection{Chapter \ref{Chapter6}: \nameref{Chapter6} }

Chapter \ref{Chapter6} discusses the application of the developed SFC×GC
instrument to the study of fatty acid profiles of various potential biodiesel
feedstocks, and shows that SFC×GC can be used to separate biodiesel from
petrodiesel in diesel fuel blends.

\subsection{Chapter \ref{Chapter7}: \nameref{Chapter7} }

Chapter \ref{Chapter7} demonstrates that the use of SFC with modifiers does not
preclude the use of the flame ionization detector when GC is used as a second
dimension.

\section{Contributions of this study}

The research presented in this thesis contributes to the field of chromatography
in the following aspects, listed in order of decreasing novelty.

\subsection{Coaxial heater cooling}

The most significant new contribution to chromatography of this project is
proving the concept of active cooling of short chromatography columns by
precisely-applied liquid carbon dioxide. To the best of our knowledge there are
no other fast temperature-programmed chromatographs with sub-ambient ramp start
temperatures and higher repetition rates.

\subsection{Coaxial heater reliability}

Previous work in our laboratories proved the concept of SFC×GC with
resistively-heated fast temperature programming \autocite{Venter2004,
Venter2006}, but the repeatability of the gas chromatography was not good enough
for practical implementation. The implementation of the coaxial heater made the
heating of the column highly reliable and repeatable, enabling of hundreds of
consecutive fast GC runs to be registered with supreme retention time stability.

\subsection{Integrated heating and sensing elements}

Resistive heating has been used in chromatography before, but to the best of our
knowledge this is the first time a coaxial heater simultaneously serves as
temperature sensing element.

\subsection{Modified SFC with FID}

Modern SFC mostly uses carbon dioxide as a mobile phase, with modifiers added.
This precludes the use of the flame ionization detector (SFC-FID), because the
signal from the organic modifier will swamp the signal from the analyte. But when a
fraction collected from a modified-SFC separation is subjected to a GC
separation the modifiers and the analytes are separated, and the signal
from the analyte can be captured while the signal from the organic modifier is ignored.
We demonstrated that modified-SFC comprehensively coupled to GC yields a space
that can be used for novel separations and reliable quantitation.

\subsection{Length of SFC column}

The low viscosity of carbon dioxide allows for the use of very long packed
columns. We were able to run SFC separations on five columns packed with
\SI{3}{\micro\metre} particles in series. There were no problems in obtaining
adequate flow using an inlet pressure of \SI{200}{\bar}.
\section{Special challenges}

The fact that the study came to a successful conclusion does not mean that
success was ever guaranteed. A few problems brought the project close to
failure and tested perseverance.

\subsection{Restrictor blocking}

The persistent blocking of the Guthrie restrictor (Section \ref{sec:Restrictor})
came as a surprise. No previous work in our laboratories have experienced it,
and our colleagues in industry who used them have also not experienced it.
Identifying the true root of the problem took patience and persistence, and lead
to the first measurement of the size of the Guthrie restrictor orifice. In the
end a linear restrictor was used, which means that we are not sure to what
degree volatility discrimination takes place.

\subsection{ADC limitations}

The resistive heater initially worked as expected at low applied power, but at
higher powers inexplicable deviations appeared. At first we handled the problem
by restricting the applied power to the range where it behaved as expected, but
that limited the heating rate of the coaxial heater. Persistent study of the
problem eventually revealed that we tried to measure a voltage that exceeded the
safe operating range of the analog-to-digital converter (ADC) device. Under
these conditions safety circuits were activated, resulting in the ADC providing
unexpected values. The problem was solved by installing an \keyword{isolation
amplifier}.


\section{Design}

Although this was a 'scientific' project, with an emphasis on demonstrating
theoretical ideas, over time it became apparent that using proper design
principles when building scientific apparatus saves time and energy in the long
run.

\subsection{Design weaknesses}

\subsubsection{Efficiency}

The design of the SFC×GC is not particularly energy-efficient, and two elements
of the design can be greatly improved. Firstly, the heated T-piece blocks are
probably not the best solution to the problem. Together they demand up to
\SI{800}{\watt} of electrical power, for which there is no significant
scientific return. Secondly, the power supply to the coaxial heater is a
\keyword{linear power supply}. In practical terms this means that the excess
power of the voltage drop between the DC power supply and the coaxial heater
input is dissipated as heat. A future design for a resistively, coaxially heated
fast gas chromatograph would be more efficient if it was powered by a
\keyword{switched-mode power supply}.

\subsubsection{Two-wire resistance measurement}

The resistive heater had a design that depended on measuring the electrical
resistance of the thin-walled stainless steel tube. The resistance was measured
by comparing the potential difference between the terminals of the coaxial
heater with the potential difference between the terminals of a reference
resistor in series with the coaxial heater. The circuit that carried the current
also measured the potential difference. Because the current was high, the
circuit had to be constructed in such a way that the coaxial heater was the only
significant resistor, and also the only resistance that changed, which meant
that care had to be taken to use heavy-gauge cable and only use soldered joints.

A future design could relieve some of these problems if the current circuit and
the measuring circuit were separated, using \keyword{four wire resistance
measurement}. In such a design the potential difference between the terminals of
the coaxial heater is measured using a circuit that connects directly to the
voltmeter, and the current through the coaxial heater is carried by a separate
circuit. The voltmeter has a high input impedance, which means that the current
the circuit carries is low and stray resistances will only have a small effect
on the measurement. Also this makes it possible to use bolted or plugged
connectors on the current-carrying circuit, which will simplify operations.

\subsubsection{Cooling control}

The fast GC temperature was poorly controlled in the \SIrange{-20}{50}{\celsius}
region of the temperature ramps, because the coaxial heater was not thermally
isolated. It was mounted in a GC oven together with the heated T-piece blocks.
The heat leaking from these uninsulated blocks accumulated in the oven, until
the oven reached a temperature of about \SI{50}{\celsius}. Once the coolant flow
stopped the temperature of the coaxial heater would immediate rise to the
temperature of the oven. Also, the coolant flow was only on/off controlled. If a
future design uses variable orifice control or \keyword{pulse width modulation
control} (PWM), the actual cooling power can be controlled and with it the
trapping temperature of the column.

\subsubsection{Intertwined operation}

The system, as built, does not allow for the separate operation of the SFC and
the GC: the SFC needs the fast GC as a detector, and the fast GC can only inject
fractions collected from the SFC. This means that it is not possible to trial a
1D separation on SFC before engaging the \twoD GC, or to optimize the GC
independently of running the SFC. Adding an optical detector to the SFC, or
writing code that will allow the manual injection of samples into the GC will
make optimizing more more flexible.

\subsubsection{General-purpose computer operating system}

The software ran on a general-purpose operating system (Microsoft Windows XP or
Microsoft Windows 7), designed for interactive use by a human operator. In such
a system there are many computing tasks running in parallel. The user is usually
not aware that many tasks are running, because the operating system (usually)
switches between tasks in less time than it takes a human to notice. The
operating system decides which task gets priority. To most humans it is at most
an irritating experience if the task they are working on gets lower priority
than another task, but when the task is controlling a machine and it doesn't get
priority, then accidents can happen. As an example of what can go wrong, a
heater was left switched on for too long because the heater control task was not
given any processor time: the operating system had given priority to a
non-critical software-updating task. This resulted in overheating and the risk
of fire. On other occasions data got corrupted because data collection tasks got
lower priority. Future designs of SFC×GC instruments can benefit from the use of
\keyword{real-time operating systems.}

\subsection{Design strengths}

\subsection{Metal protection for column}

Sheathing a capillary GC column in a stainless steel tube proved to be a very
good idea. Combining the multiple roles of heating element, temperature sensing
element, conduit for cryogenic coolant, and column protector into a single
object makes for a robust, efficient design element that could find multiple
applications.

\subsubsection{Time decoupling}

The use of stopped-flow SFC decouples the \oneD time from the \twoD time. This
means that the \twoD run time can be longer than the modulation period, but it
also means that it is possible to vary the modulation period during an SFC×GC
run. This would allow one to speed up analysis by collecting fewer fractions
where less information would be expected, or to get more detail in interesting
portions of the \oneD chromatogram.


\section{Aspects to be addressed} 

Since the focus of this study was the development of practical instrumentation,
some aspects of good chromatography and good design were overlooked.

\subsection{Lack of optimization}

None of the flow rates or heating rates were optimized, so the results merely
prove that it is possible to do chromatography using SFC×GC but do not provide
any figures of merit.

\subsection{Effect of modifiers}

Although we demonstrated that modifiers can be added to the SFC mobile phase
when running SFC×GC-FID, we did not offer any demonstration of separations that
require modifiers.

\subsection{Lack of quantification}

It was demonstrated that the SFC×GC can separate compounds, but not that they
can be quantified, for example by setting up calibration curves. But the
technology for quantification from 2D chromatograms is mature, and we don't
believe that quantification by SFC×GC should prove any obstacles.

\subsection{Generalized voltage ranges}

When designing the coaxial heater power electronics and control system, the
choice of material, its dimensions and the required power matched well with the
chosen analog-to-digital converter, and made it possible to use a remarkably
simple electronic control circuit. A different resistive heater system might
require more complex circuitry to match all the components, designed by an
experienced electronic engineer using the principles established by this
project.

\section{Future work}

On the successful operation of the SFC×GC we look forward to applications that
can benefit from the increased separation space and highly structured
information provided by SFC×GC.


\subsection{Biodiesel impurities}

SFC×GC was shown to be able to separate the main components of biodiesel and
diesel fuel blends. Future work can include the determination of impurities, in
particular free glycerol, glycerides and free fatty acids.

\subsection{Analytical applications}

Knowing that the strength of SFG×GC lies in the orthogonality of the two
separations, it would be worth doing an exhaustive literature search to find
novel or forgotten normal-phase LC separations that effectively separate
compounds by functional group. Such methods could be adapted to SFC×GC to expand
the analysis of complex mixtures.

\section{Summary}

The high repeatability of fast temperature ramps afforded by a
resistively-heated coaxial heating of short GC columns, combined with the high
repetition rate afforded by coaxial cooling of column, applied to the analysis
of fractions of compounds separated by the selective group-type separation of
SFC, creates a powerful chromatograph that can find a use in many different
aspects of quality control in the biodiesel industry.

