% Chapter 3

\begin{savequote}[60mm]
The nice thing about standards is that you have so many to choose from.
\qauthor{Andrew S. Tanenbaum}
\end{savequote}


\chapter[Biodiesel standards]{Biodiesel, technical standards and chemical analysis} % Main chapter title

\label{Chapter3} % For referencing this chapter elsewhere, use \ref{Chapter3}

The fact that vegetable oil can be used to fuel Diesel engines has been known
since the earliest days. Rudolf Diesel himself had exhibited an engine at the
Paris Exhibition in 1998 \todo{autocite Diesel peanut oil} that ran on peanut
oil. But the development of the petroleum industry late in the 19th century
ensured an ample supply of fossil fuel for these engines.

The development of the diesel engine happened in parallel with the developing
diesel fuel, for, as Cummins said \autocite{Cummins1989}``\ldots we must never
forget that \textit{an engine and the fuel it consumes are inseparable
partners}; the one cannot progress without the full cooperation of the other.''
The first invention of an engine presupposes a supply of fuel, but the drive
towards lower costs and higher efficiency then opens up the quest for more
fuels. Technically, engine builders would then have to test each new fuel in
their engines and approve of it. But fuel suppliers would like to see their
fuels used in as many engines as possible. This convergence of interests gives
rise to the establishment of \textit{technical standards}.

A technical standard or just \textit{standard} is a ``document, established by
consensus and approved by a recognized body, that provides, for common and
repeated use, rules, guidelines or characteristics for activities or their
results, aimed at the achievement of the optimum degree of order in a given
context'' \autocite{Hatto2010}. 

Standards are not mandatory, nor do they provide the `best' way of doing
something. The great strength of standards are their reliability.
``Standards exist principally to provide a reliable basis on which common
expectations can be shared regarding specific characteristics of a product,
service or process'' \autocite{BSI2016}.

Standards are published by \textit{standards organizations}, which might
be national or international in character, or might be established to serve a
certain industry. Standards organizations are often known by their
abbreviations, and a few of them are listed in Table
\ref{tab:StandardsOrganizations}. The authors of these standards documents are
usually \keyword{technical committees}, comprised of individuals from a wide
variety of stakeholder organizations, who work towards consensus.

\begin{table}
	\caption{A few well-known standards organizations}
	\label{tab:StandardsOrganizations}
	\centering
	\begin{tabular}{l l l}
	\toprule
	\tabhead{Abbreviation} & \tabhead{Name} & \tabhead{Country of origin} 		\\
	\midrule
	SABS 	& South African Bureau of Standards 		& South Africa	\\
	ISO		& International Organization for Standardization & International \\
	CEN 	& European Committee for Standardization 	& Europe		\\
	ASTM 	& ASTM International 						& USA			\\
	BSI	 	& British Standards Institution 			& UK			\\
	IEC 	& International Electrotechnical Commission & International \\
	DIN 	& German Institute for Standardization 		& Germany 		\\
	ANSI 	& American National Standards Institute 	& USA 			\\
	UL 		& Underwriter's Laboratory 					& USA 			\\
	ITU 	&International Telecommunication Union		& International \\
	\bottomrule\\
	\end{tabular}
\end{table}

Standards have a unique publication system. They are not published by publishing
houses, but by the standards organizations themselves, who sell the documents
directly to end users. Each standard is usually also better known by its number
than by its title. For example, if I were to mention the document entitled
``Quality management systems — Requirements'' few people would know that I'm
talking about the well-known quality system standard usually known as ISO 9001.

The South African national standards body is the South African Bureau of
Standards (SABS), which was established by an act of parliament,
 Standards Act, 1945 (Act No. 24 of 1945)), as amended by the Standards Act, Act
 No. 8 of 2008 \autocite{Act8-2008}. The SABS issues South African National
 Standards.

Standards organizations not only write standards, they might also
\keyword{adopt} them. Adoption happens when a suitable standard has already been
issued by another standards organization. Standards very often refer to other
standards, and standards are often based on published research. While standards
are not mandatory, some legislation might refer to 

The desire for engine designers for access to a reliable fuel and for fuel
suppliers to have the largest possible market lead them to cooperate in the
development of standards for fuels. In South Africa the relevant standard for
diesel fuel is SANS 342, and the US equivalent is ASTM D975.

Biodiesel is chemically very different from diesel, and therefore the technical
standards of biodiesel needs to be different from the standards for diesel.

\todo{Normative vs informative}

\section{SANS 1935: An overview}

The current South African standard applicable to biodiesel is ``South African
National Standard 1935 Automotive biodiesel — Fatty Acid Methyl Esters (FAME)
for diesel engines — Requirements and test methods.'' \autocite{SANS1935} This
is an adoption of the European Committee for Standardization's (CEN) standard EN
14214 \autocite{EN14214}. In the USA is the equivalent standard is ASTM D 7651,
which is largely similar but of different heritage.

SANS 1935 consists of a 18 pages. The first two pages are unnumbered: for the
purposes of this discussion they will be numbered in small Roman numerals.

\begin{description}


\item[p(i)]{The first page is a title page, following the usual format for SABS
standards. The top line of the page contains the ISBN (978-0-626-26349-2), and
in large type the standard number (SANS 1935:2011). Then follows in capital
letters ``South African National Standards'', and below that the title
``Automotive biodiesel --- FFatty Acid Methyl Esters (FAME) for diesel engines
--- Requirements and test methods''. At the bottom edge of the page we find the
SABS logo and some contact information.}

\item[p(ii)]{The second page is an informational page. It starts with a table of
changes, which was at the time of writing still empty. Then follows a foreword,
in which the committee who approved the standard is acknowledged (National
Committee SABS SC 1018A). It also gives the date of publication (December 2011)
and states that it supersedes SANS 1935:2004. Then there's very significant
line, which states that the standard is referenced in the Petroleum Products Act
\autocite{Act120-1997}. }
	
\item[p1]{Contains the Table of Contents} 

\item[p2]{Is left blank}

\item[p3]{Paragraph 1: This paragraph describes the scope of the standard, which
is that it applies to biodiesel as an automotive fuel.} 

\item[p3]{Paragraph 2:
This paragraph lists all the normative standards required to comply with to
comply with SANS 1935. Thirty-five standards are listed.} 

\item[p4]{Continues the list of normative references}

\item[p5]{Paragraph 3: This paragraph contains a list of definitions. The most
important one is this: ``biodiesel [is a] fuel comprised of methyl esters of
long chain fatty acids derived from vegetable oils.'' This is a very specific
description of biodiesel. It eliminates animal fats as a source of fatty acids,
and it excludes methyl esters. But a note reads ``Consideration for the
inclusion of ethyl esters, animal fats and C8 – C12 carbon chains should be
taken later.'' The significance of ethyl esters is that methyl esters are not
usually carbon neutral. The methanol used in the transesterification reaction is
usually obtained from the petrochemical industry, whereas ethanol from
fermentation (See Section \ref{sec:BioEthanol}) could be carbon neutral. The
definition also rules out hydrotreated vegetable oil or Fischer-Tropsch diesel
derived from biomass.}

\item[p5]{Paragraph 4: This paragraph lists requirements} 

\item[p5]{Paragraph 4.1: Discusses general requirements. According to these
requirements biodiesel is a homogeneous liquid, free of adulterants or
colourants, to which additives might be added. It prescribes testing for
oxidative stability and cold-flow properties}

\item[p6]{Paragraph 4.2: This paragraph is about physical and chemical
properties and states that biodiesel shall comply to the requirements of Table
1}

\item[p7]{Paragraph 4.3: This paragraph concerns methods of test. It states that
test methods shall be  one of the test methods listed in Table 1.}

\item[p7]{Paragraph 4.4 concerns disputes. It comes into effect when there is a
dispute about product quality between, say, a biodiesel manufacturer and a
biodiesel distributor. The contents of this paragraph prescribes which reference
method shall be used.}

\item[p7-p8]{These pages contain Table 1. The table has three columns. The first
column names a property, the second column states the numerical values it should
comply to, and the third column specify which test metod should be used to
determine that value.}

\item[p9]{Paragraph 5 describes a few simple rules for packing and marking biodiesel.}

\item[p10-p11]{Annex A describes a method for the calculation of iodine value,
which might be used instead of a measurement.}

\item[p12]{Annex B prescribes how samples for testing must be taken.}

\item[p13]{Annex C gives a list of values for calculating precision.}

\item[p14]{Annex D provides an approved method for correcting density
measurements. The prescribed tests requires density to be measured at
\SI{15}{\celsius}, which be might inconvenient. Instead, a different test may be
made at a more convenient temperature, and a correction applied. }

\item[p15]{Annex E is informative and recommends the implementation of quality
management systems. It is followed by a bibliography.}

\item[p16]{The final page contains information about the SABS and its services.}

\end{description} 

\section{SANS 1935: Properties, requirements and methods.}

On examining SANS 1935, it quickly becomes clear that the core of the document
is Table 1. This table has three columns. The first column lists a
\keyword{property}, the second column specifies a numerical value the property
has to conform to, the \keyword{requirement}, and the third column prescribes
the \keyword{test method} that must be used to obtain the value.

In the following discussion each of the requirements will be discussed, in order
of increasing relevance to this thesis, grouped by method of determination.

\section{SANS 1935: Physical methods}

\subsection{Density}

The density of biodiesel is required to be
\SIrange{860}{900}{\kilogram\per\cubic\metre} at at temperature of
\SI{15}{\celsius} The metods that can be used is described in ISO 3675 and ISO
12185. If ISO 3675 is used at a temperature other than the specified one, a
temperature correction is applied, as described in Annex D.

ISO 12185 prescribes the measurement of density by using an electronic
instrument known as the oscillating tube density meter. This measures the
density of a liquid by measuring the frequency of a tube filled with the liquid
that's oscillating freely. The frequency depends on the weight of the filled
tube, and therefore on the density of the liquid. These devices are easy to use
and very accurate. They are temperature controlled

ISO 3675 prescribes the use of a hydrometer, which is an instrument that
measures density by measuring the buoyant force on a floating indicator.
According to the law of Archimedes, the buoyant force on a body immersed in a
liquid is equal to the weight of the displaced liquid. If the liquid is denser,
the force is greater. Therefore, in a denser liquid a floating indicator would
float with more of the indicator above the surface of the liquid

\subsection{Kinematic viscosity}

The viscosity of a fluid is a measure of its resistance to flow when a force is
applied to it. Kinematic viscosity is the resistance of the flow to the force of
gravity. This flow of course depends on the density of the liquid, so that It is
important that the fuel for a diesel engine have the right viscosity, because

SANS 1935 requires that kinematic viscosity be in the \SIrange{3.5}{5.0}. The
measurement is specified by ISO 3104, which is a capillary method.

\section{SANS 1935: Aggregate properties: Specialized instrumentation}

Some of the requirements specified in SANS 1935 are not values that have direct
correspondence to physical or chemical quantities usually used in science. 

\subsection{Cetane number}

The cetane number is a number that indicates the ease of ignition of a fuel in a
diesel engine. A higher number indicates easier ignition. SANS 1935 prescribes
the test method of ISO 5165.

Cetane is a synonym for hexadecane, and is a liquid compound that ignites easily
when injected into a diesel engine. In contrast, its highly branched isomer
2,2,4,4,6,8,8-heptamethylnonane (also called isocetane) ignites less easily. The
intuitively-understood ``ease of combustion" can be quantified as the
\keyword{ignition delay}, the time between the moment when the fuel is injected
in to the engine and moment the ignition starts. A mixture of the two compounds
will have an ignition delay somewhere between that of the two compounds.

The cetane number is obtained by measuring the ignition delay in a  highly
specialized test engine. This is a single-piston, four-stroke engine that has a
combustion chamber of variable volume, which gives it a variable compression
ratio. While running on the fuel under test, the compression ratio is changed
until a prescribed ignition delay is achieved. Then the cetane/isocetane mixture
that will give the same ignition delay at that compression ratio is found. If
the fuel under test has the same ignition delay as a pure cetane, it is assigned
the cetane number of 100. If it has an ignition delay that is the same as
isocetane, then the cetane number 0 is assigned, and if it has the same ignition
delay as a 50:50 cetane:isocetane mixture, then it is assigned the cetane number
of 50.

Cetane number gives limited insight into the chemical properties of the fuel, or
indeed, engine performance. But it is a trusted measure of the suitability of a
fuel for use in a diesel engine. 

\subsection{Flash point}

Fuels are, naturally, flammable, and different fuels are flammable to different
degrees. A fuel's \keyword{flash point} can be used to quantify its degree of
flammability. The flash point is the temperature at which a fuel's vapour will
ignite when it comes in contact with a flame. 

The measurement of flash point is not primarily concerned with the performance
of the fuel inside the engine, but is important to know because it determines
how the fuel can be safely be handled during transport and storage
\autocite{WFCC2009}.

The flash points of liquids are established by standardized tests in automated
apparatus.

\subsection{Oxidation stability}

Fossil fuels for diesel engines are largely composed of alkanes, which are
chemically very inert. This means that they can be stored for long periods
without significant degradation.

Biodiesel, by contrast, are by definition (\autocite[Paragraph 4.1.1]{SANS1935})
mostly composed of fatty acids by definition, which are not inherently stable.
This instability has been known since antiquity as fatty food going
\keyword{rancid}. In particular, double bonds in the fatty acids are liable to
reactions with atmospheric oxygen \autocite(Velasco2010). These are free radical
reactions, illustrated in Figure \ref{fig:RancidRadical}.

\begin{figure}
\centering
\includegraphics[width=\textwidth]{Figures/1281px-Lipid_peroxidation.svg.png}
\decoRule

\caption[The peroxidation of lipids.]{The peroxidation of lipids. By Tim
Vickers, after \autocite{Young2001}. - w:Image:Lipid peroxidation v2.png, Public
Domain, https://commons.wikimedia.org/w/index.php?curid=1728531}

\label{fig:RancidRadical}
\end{figure}


\todo{The reaction products are short-chain fatty acids, soaps, sludges.}

The oxidative stability is measured by the Rancimat, an automated instrument. It
works by bubbling a stream of hot air through a sample of biodiesel. This air is
then passed through a conductivity measuring cell filled with distilled water.
The conductivity of the water increases when volatile acids dissolves in it. At
the beginning of the test period there are very few acids dissolved swept into
the water, so the conductivity remains low. As the sample oxidizes and produce
acids, the conductivity of the water slowly increases. But the oxidation
reaction is \keyword{autocatalytic}, which means that the products of the
oxidation reaction accelerates the oxidation. The result of the autocatalysis is
that on the conductivity curve there is a sudden increase in conductivity. The
time taken until this point is reached is called the \keyword{induction period}.

SANS 1935 requires a minimum induction period of 6 minutes at 110
\SI{110}{\celsius}. Paragraph 4.1.2 permits the addition of antioxidants.

\subsection{Cold filter plugging point}

The flow of liquids depend on the temperatures. Most obviously, their densities
depend on the temperature. But liquids can freeze, which will also influence
their flows. Inside an engine at operating temperature the temperature of the
fuel will be high enough, but the biodiesel needs to be pumped to the engine
from the fuel tank, passing through a fuel filter on the way. If crystals of
fuel form at low temperatures, these crystals might plug the pores in the
filter, which could lead to fuel starvation in the engine. It is therefore
important that the biodiesel must be able to be pumped through the engine's fuel
filter at all expected temperatures.

For this purpose SANS 1935 specifies a \keyword{cold filter plugging point}. It
is the lowest temperature at which a given volume of biodiesel still passes
through a standardized filtration device in a specified time when cooled under
certain conditions. This is an entirely practical test: no fundamental
physical property is measured.

\section{SANS 1935: Chemical properties: Classical determination}

\subsection{Sulfated ash}

Ash is the solids remaining after the complete combustion of a fuel. This means

\keyword{Sulfated ash} is the residue remaining after the lubricating oil sample has been
carbonized, and the residue subsequently treated with sulfuric acid and heated
to constant mass.



\subsection{Carbon residue}

\subsection{Total contamination}

\section{SANS 1935: Chemical properties: Electrochemical determination}

\section{SANS 1935: Chemical properties: Spectroscopic determination}

\section{SANS 1935: Chemical properties: Chromatographic determination}

\section{Petroleum Products Act}

SANS 1935 is referred to in the Petroleum Products act, which makes it illegal. 

\section{Fast Gas Chromatography}

\todo{autocite Blumberg 1997}