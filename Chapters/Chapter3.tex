% Chapter 3

\begin{savequote}[60mm]
The nice thing about standards is that you have so many to choose from.
\qauthor{Andrew S. Tanenbaum}
\end{savequote}


\chapter[Biodiesel standards]{Biodiesel, technical standards and chemical analysis} % Main chapter title

\label{Chapter3} % For referencing this chapter elsewhere, use \ref{Chapter3}

The fact that vegetable oil can be used to fuel Diesel engines has been known
since the earliest days. Rudolf Diesel himself had exhibited an engine at the
Paris Exhibition in 1998 \todo{autocite Diesel peanut oil} that ran on peanut
oil. But the development of the petroleum industry late in the 19th century
ensured an ample supply of fossil fuel for these engines.

The development of the diesel engine happened in parallel with the developing
diesel fuel, for, as Cummins said ``\ldots we must never forget that \textit{an
engine and the fuel it consumes are inseparable partners}; the one cannot
progress without the full cooperation of the other.'' \autocite{Cummins1989} The
first invention of an engine presupposes a supply of fuel, but the
variation-selection process that searches for lower costs and higher efficiency
then opens up the quest for more fuels. If a possible fuel is found, meticulous
engine builders would then have to test each new fuel in their engines and
approve of it. But fuel suppliers would like to see their fuels used in as many
engines as possible. This convergence of interests gives rise to the
establishment of \textit{technical standards}. Technical standards allows engine
builders to develop engines that will run on any fuel with certain agreed-upon
standard qualities, and fuel producers can produce fuels knowing that they will
work on any engine designed for that fuel. 

A technical standard or just \textit{standard} is a ``document, established by
consensus and approved by a recognized body, that provides, for common and
repeated use, rules, guidelines or characteristics for activities or their
results, aimed at the achievement of the optimum degree of order in a given
context'' \autocite{Hatto2010}. 

Standards are not mandatory, nor do they provide the `best' way of doing
something. The great strength of standards are their reliability.
``Standards exist principally to provide a reliable basis on which common
expectations can be shared regarding specific characteristics of a product,
service or process'' \autocite{BSI2016}.

Standards are published by \textit{standards organizations}, which might
be national or international in character, or might be established to serve a
certain industry. Standards organizations are often known by their
abbreviations, and a few of them are listed in Table
\ref{tab:StandardsOrganizations}. The authors of these standards documents are
usually \keyword{technical committees}, comprised of individuals from a wide
variety of stakeholder organizations, who work towards consensus.

\begin{table}
	\caption{A few well-known standards organizations}
	\label{tab:StandardsOrganizations}
	\centering
	\begin{tabular}{l l l}
	\toprule
	\tabhead{Abbreviation} & \tabhead{Name} & \tabhead{Country of origin} 		\\
	\midrule
	SABS 	& South African Bureau of Standards 		& South Africa	\\
	ISO		& International Organization for Standardization & International \\
	CEN 	& European Committee for Standardization 	& Europe		\\
	ASTM 	& ASTM International 						& USA			\\
	BSI	 	& British Standards Institution 			& UK			\\
	IEC 	& International Electrotechnical Commission & International \\
	DIN 	& German Institute for Standardization 		& Germany 		\\
	ANSI 	& American National Standards Institute 	& USA 			\\
	UL 		& Underwriter's Laboratory 					& USA 			\\
	ITU 	&International Telecommunication Union		& International \\
	\bottomrule\\
	\end{tabular}
\end{table}

Standards have a unique publication system. They are not published by publishing
houses, but by the standards organizations themselves, who sell the documents
directly to end users. Each standard is usually also better known by its number
than by its title. For example, if I were to mention the document entitled
``Quality management systems — Requirements'' few people would know that I'm
talking about the well-known quality system standard usually known as ISO 9001.

The South African national standards body is the South African Bureau of
Standards (SABS), which was established by an act of parliament,
 Standards Act, 1945 (Act No. 24 of 1945)), as amended by the Standards Act, Act
 No. 8 of 2008 \autocite{Act8-2008}. The SABS issues South African National
 Standards.

Standards organizations not only write standards, they might also
\keyword{adopt} them. Adoption happens when a suitable standard has already been
issued by another standards organization. Standards very often refer to other
standards, and standards are often based on published research. While standards
are not mandatory, some legislation might refer to 

The desire for engine designers for access to a reliable fuel and for fuel
suppliers to have the largest possible market lead them to cooperate in the
development of standards for fuels. In South Africa the relevant standard for
diesel fuel is SANS 342, and the US equivalent is ASTM D975.

Biodiesel is chemically very different from diesel, and therefore the technical
standards of biodiesel needs to be different from the standards for diesel.

\todo{Normative vs informative}

\section{SANS 1935: An overview}

The current South African standard applicable to biodiesel is ``South African
National Standard 1935 Automotive biodiesel — Fatty Acid Methyl Esters (FAME)
for diesel engines — Requirements and test methods.'' \autocite{SANS1935} This
is an adoption of the European Committee for Standardization's (CEN) standard EN
14214 \autocite{EN14214}. In the USA is the equivalent standard is ASTM D 7651,
which is largely similar but of different heritage.

SANS 1935 consists of a 18 pages. The first two pages are unnumbered: for the
purposes of this discussion they will be numbered in small Roman numerals.

\begin{description}


\item[p(i)]{The first page is a title page, following the usual format for SABS
standards. The top line of the page contains the ISBN (978-0-626-26349-2), and
in large type the standard number (SANS 1935:2011). Then follows in capital
letters ``South African National Standards'', and below that the title
``Automotive biodiesel --- FFatty Acid Methyl Esters (FAME) for diesel engines
--- Requirements and test methods''. At the bottom edge of the page we find the
SABS logo and some contact information.}

\item[p(ii)]{The second page is an informational page. It starts with a table of
changes, which was at the time of writing still empty. Then follows a foreword,
in which the committee who approved the standard is acknowledged (National
Committee SABS SC 1018A). It also gives the date of publication (December 2011)
and states that it supersedes SANS 1935:2004. Then there's very significant
line, which states that the standard is referenced in the Petroleum Products Act
\autocite{Act120-1997}. }
	
\item[p1]{Contains the Table of Contents} 

\item[p2]{Is left blank}

\item[p3]{Paragraph 1: This paragraph describes the scope of the standard, which
is that it applies to biodiesel as an automotive fuel.} 

\item[p3]{Paragraph 2:
This paragraph lists all the normative standards required to comply with to
comply with SANS 1935. Thirty-five standards are listed.} 

\item[p4]{Continues the list of normative references}

\item[p5]{Paragraph 3: This paragraph contains a list of definitions. The most
important one is this: ``biodiesel [is a] fuel comprised of methyl esters of
long chain fatty acids derived from vegetable oils.'' This is a very specific
description of biodiesel. It eliminates animal fats as a source of fatty acids,
and it excludes methyl esters. But a note reads ``Consideration for the
inclusion of ethyl esters, animal fats and C8 – C12 carbon chains should be
taken later.'' The significance of ethyl esters is that methyl esters are not
usually carbon neutral. The methanol used in the transesterification reaction is
usually obtained from the petrochemical industry, whereas ethanol from
fermentation (See Section \ref{sec:BioEthanol}) could be carbon neutral. The
definition also rules out hydrotreated vegetable oil (see Section
\ref{sec:GreenDiesel}) or biomass-derived Fischer-Tropsch diesel (see Section
\ref{sec:FT}.}

\item[p5]{Paragraph 4: This paragraph lists requirements} 

\item[p5]{Paragraph 4.1 discusses general requirements. According to these
requirements biodiesel is a homogeneous liquid, free of adulterants or
contaminants, to which additives might be added. It provides details regarding
testing for oxidative stability and cold-flow properties}

\item[p6]{Paragraph 4.2: This paragraph is about physical and chemical
properties and states that biodiesel shall comply to the requirements of Table
1}

\item[p7]{Paragraph 4.3: This paragraph concerns methods of test. It states that
test methods shall be  one of the test methods listed in Table 1.}

\item[p7]{Paragraph 4.4 concerns disputes. It comes into effect when there is a
dispute about product quality between, say, a biodiesel manufacturer and a
biodiesel distributor. The contents of this paragraph prescribes which reference
method shall be used.}

\item[p7-p8]{These pages contain Table 1. The table has three columns. The first
column names a property, the second column states the numerical values it should
comply to, and the third column specify which test metod should be used to
determine that value.}

\item[p9]{Paragraph 5 describes a few simple rules for packing and marking biodiesel.}

\item[p10-p11]{Annex A describes a method for the calculation of iodine value,
from chromatographic data. This calculation might be used instead of a
measurement.}

\item[p12]{Annex B prescribes how samples for testing must be taken.}

\item[p13]{Annex C gives a list of values for calculating precision.}

\item[p14]{Annex D provides an approved method for correcting density
measurements. The prescribed tests requires density to be measured at
\SI{15}{\celsius}, which be might inconvenient. Instead, a different test may be
made at a more convenient temperature, and a correction applied. }

\item[p15]{Annex E is informative and recommends the implementation of quality
management systems. It is followed by a bibliography.}

\item[p16]{The final page contains information about the SABS and its services.}

\end{description} 

\section{SANS 1935: Properties, requirements and methods.}

On examining SANS 1935, it quickly becomes clear that the core of the document
is Table 1. This table has three columns. The first column lists a
\keyword{property}, the second column specifies a numerical value the property
has to conform to, the \keyword{requirement}, and the third column prescribes
the \keyword{test method} that must be used to obtain the value.

In the following discussion each of the requirements will be discussed, in order
of increasing relevance to this thesis, grouped by method of determination.

\section{SANS 1935: Physical methods}

\subsection{Density}

The density of biodiesel is required to be
\SIrange{860}{900}{\kilogram\per\cubic\metre} at at temperature of
\SI{15}{\celsius} The metods that can be used is described in ISO 3675 and ISO
12185. If ISO 3675 is used at a temperature other than the specified one, a
temperature correction is applied, as described in Annex D.

ISO 12185 prescribes the measurement of density by using an electronic
instrument known as the oscillating tube density meter. This measures the
density of a liquid by measuring the frequency of a tube filled with the liquid
that's oscillating freely. The frequency depends on the weight of the filled
tube, and therefore on the density of the liquid. These devices are easy to use
and very accurate. They are temperature controlled

ISO 3675 prescribes the use of a hydrometer, which is an instrument that
measures density by measuring the buoyant force on a floating indicator.
According to the law of Archimedes, the buoyant force on a body immersed in a
liquid is equal to the weight of the displaced liquid. If the liquid is denser,
the force is greater. Therefore, in a denser liquid a floating indicator would
float with more of the indicator above the surface of the liquid

\subsection{Kinematic viscosity}

The viscosity of a fluid is a measure of its resistance to flow when a force is
applied to it. Kinematic viscosity is the resistance of the flow to the force of
gravity. This flow of course depends on the density of the liquid, so that It is
important that the fuel for a diesel engine have the right viscosity, because

SANS 1935 requires that kinematic viscosity be in the \SIrange{3.5}{5.0}. The
measurement is specified by ISO 3104, which is a capillary method.

\section{SANS 1935: Aggregate properties: Specialized instrumentation}

Some of the requirements specified in SANS 1935 are not values that have direct
correspondence to physical or chemical quantities usually used in science. 

\subsection{Cetane number}

The cetane number is a number that indicates the ease of ignition of a fuel in a
diesel engine. A higher number indicates easier ignition. SANS 1935 prescribes
the test method of ISO 5165.

Cetane is a synonym for hexadecane, and is a liquid compound that ignites easily
when injected into a diesel engine. In contrast, its highly branched isomer
2,2,4,4,6,8,8-heptamethylnonane (also called isocetane) ignites less easily. The
intuitively-understood ``ease of combustion" can be quantified as the
\keyword{ignition delay}, the time between the moment when the fuel is injected
in to the engine and moment the ignition starts. A mixture of the two compounds
will have an ignition delay somewhere between that of the two compounds.

The cetane number is obtained by measuring the ignition delay in a  highly
specialized test engine. This is a single-piston, four-stroke engine that has a
combustion chamber of variable volume, which gives it a variable compression
ratio. While running on the fuel under test, the compression ratio is changed
until a prescribed ignition delay is achieved. Then the cetane/isocetane mixture
that will give the same ignition delay at that compression ratio is found. If
the fuel under test has the same ignition delay as a pure cetane, it is assigned
the cetane number of 100, and if its ignition delay is the same as isocetane,
then the cetane number 0 is assigned. Ignition delays between two extremes
assign a proportional cetane number to the fuel.

Cetane number gives limited insight into the chemical properties of the fuel, or
indeed, engine performance. But it is a trusted measure of the suitability of a
fuel for use in a diesel engine. 

\subsection{Flash point}

Fuels are, naturally, flammable, and different fuels are flammable to different
degrees. A fuel's \keyword{flash point} can be used to quantify its degree of
flammability. The flash point is the temperature at which a fuel's vapour will
ignite when it comes in contact with a flame. 

The measurement of flash point is not primarily concerned with the performance
of the fuel inside the engine, but is important to know because it determines
how the fuel can be safely be handled during transport and storage
\autocite{WFCC2009}.

The flash points of liquids are established by standardized tests in automated
apparatus.

\subsection{Oxidation stability}
\label{sec:Rancimat}

Fossil fuels for diesel engines are largely composed of alkanes, which are
chemically very inert. This means that they can be stored for long periods
without significant degradation.

Biodiesel, by contrast, are by definition (\autocite[Paragraph 4.1.1]{SANS1935})
mostly composed of fatty acids, which are not inherently stable.
This instability has been known since antiquity as fatty food going
\keyword{rancid}. In particular, double bonds in the fatty acids are liable to
reactions with atmospheric oxygen \autocite{Velasco2010}. These are free radical
reactions, illustrated in Figure \ref{fig:RancidRadical}.

\begin{figure}
\centering
\includegraphics[width=\textwidth]{Figures/1281px-Lipid_peroxidation.png}
\decoRule

\caption[The peroxidation of lipids.]{The peroxidation of lipids. By Tim
Vickers, after \autocite{Young2001}. - w:Image:Lipid peroxidation v2.png, Public
Domain, https://commons.wikimedia.org/w/index.php?curid=1728531}

\label{fig:RancidRadical}
\end{figure}


\todo{The reaction products are short-chain fatty acids, soaps, sludges.}

The oxidative stability is measured by the Rancimat, an automated instrument. It
works by bubbling a stream of hot air through a sample of biodiesel. This air is
then passed through a conductivity measuring cell filled with distilled water.
The conductivity of the water increases when volatile acids dissolves in it. At
the beginning of the test period there are very few acids swept into the water,
so the conductivity remains low. As the sample oxidizes and produce acids, the
conductivity of the water slowly increases. But the oxidation reaction is
\keyword{autocatalytic}, which means that the products of the oxidation reaction
accelerates the oxidation. The result of the autocatalysis is that on the
conductivity curve there is a sudden increase in conductivity. The time taken
until this point is reached is called the \keyword{induction period}.

SANS 1935 requires a minimum induction period of 6 minutes at
\SI{110}{\celsius}. Paragraph 4.1.2 permits the addition of antioxidants.

\subsection{Cold filter plugging point}

The flow of liquids depend on the temperature. Most obviously, their viscosities
depend on the temperature. But liquids can freeze, which will also influence
their flows. The operating temperature of a diesel engine is so high that the
temperature of the fuel will be high enough to guarantee adequate flow, but the
biodiesel needs to be pumped to the engine from the fuel tank, passing through a
fuel filter on the way. If crystals of fuel form at low temperatures, these
crystals might plug the pores in the filter, which could lead to fuel starvation
in the engine. It is therefore important that the biodiesel must be able to be
pumped through the engine's fuel filter at all expected temperatures.

For this purpose SANS 1935 specifies a \keyword{cold filter plugging point}. It
is the lowest temperature at which a given volume of biodiesel still passes
through a standardized filtration device in a specified time when cooled under
certain conditions. This is an entirely practical test: no fundamental
physical property is measured.

\subsection{Copper strip corrosion}

Storage systems and engines parts are made of metals, and therefore susceptible
to corrosion under adverse conditions. Chemical conditions that make corrosion
more likely include acidity, water, and oxygen, but corrosion is a complex
process, so it is very difficult to predict which combination of factors will
result in unacceptable corrosion.

SANS 1935 prescribes ISO 2160 as the test for corrosiveness. In this test a
strip of pure copper metal is polished and then immersed in a sample of the
biodiesel at \SI{50}{\celsius} for \SI{3}{\hour}. The degree of corrosion is
judged by visually comparing it to a standard card, and then assigning it to a
class.

\section{SANS 1935: Chemical properties: Classical determination}

\subsection{Sulphated ash}

Ash is the solids remaining after the complete combustion of a fuel. This is
measure of the solids that will remain after use. Ash might originate from
suspended solids, soluble metallic soaps, and residual biodiesel catalyst. These
metals

It would not be practical to actually combust the sample and weigh the residue,
because the ash might be carried away, either as volatile species or as a finely
divided aerosol. Therefore the sample is oxidized using sulphuric acid, and then
heated to drive off the sulphuric acid. What remains is metal oxides, which can
be weighed.

\subsection{Carbon residue}

When a mixture of organic compounds is heated under conditions of low oxygen, it
can form coke. Coke is a material that is practically pure carbon. Pure carbon
will combust (\todo{delta H}), but the reaction is kinetically limited, so it
will only combust slowly under conditions of high temperature and a large excess
of oxygen. If a fuel tends to form coke inside an engine, it can cause problems
in operation. Vegetable oil as diesel fuel, for example, tend to form coke on
the injectors \autocite{Walt1982}. This coking can cause problems with
injection, which would affect engine performance.

The test method prescribed for testing for carbon residue is ISO 10370. This
involves heating a sample of biodiesel to high temperature in a crucible in air,
using standardized apparatus. Most of the sample burns off, leaving a residue.
The weight of the residue is determined.

Carbon residue is not a scientific measure: it has been found to correlate with
properties of oils in the petroleum industry.

\subsection{Total contamination}

Ideally, biodiesel should be a homogeneous liquid. Some undissolved material can
be tolerated, but too much can plug filters.

SANS 1935 specifies the test SANS 52662, which is synonymous with EN 12662. In
this test total contamination is determined by filtering a sample of the
biodiesel being tested on glass-fibre filter, and determining the mass of the
retained material.

\subsection{Acid value}

The end products of oxidation of biodiesel (see Section\ref{sec:Rancimat})
includes free organic acids, so the acidity of biodiesel is a good indicator of
its quality. Measuring the acidity gives an indication of how much the biodiesel
has already oxidized, whereas the oxidation stability (measured by Rancimat)
indicates how well the biodiesel will withstand oxidation on storage.

SANS 1935 specifies the test described in SANS 54104, which is equivalent to EN
1410414. This test is a titration of a sample of biodiesel dissolved in a
mixture solvents with alcoholic KOH. A glass pH electrode is used to follow the
titration.

\subsection{Iodine value}

As discussed above (see Section \ref{sec:Rancimat}) the oxidation of biodiesel
is a major quality concern. The tendency of biodiesel to oxidize correlates with
the number of double bonds in the fatty acids, or their \keyword{degree of
unsaturation}.

% t AOCS method Cd 1d-92 using cyclohexane-acetic acid (also used by the
% European standard EN 14111) as solvent system are based on the Wijs solution.

The degree of unsaturation has long been measured by the \keyword{iodine value},
dating from 1884 \autocite{Knothe2007}. Halogens will rapidly add to double
bonds, so that when a mixture of fatty acids is treated with a known excess of
iodating reagent, the remaining iodine can be titrated to determine the amount
of iodine absorbed by the fatty acids. The iodine value is the mass of iodine
absorbed by 100 mass units of fat or oil.

The relevance of including iodine value in biodiesel standards has been
questioned \autocite{Knothe2002}, because all the information regarding
unsaturation of the fatty acids in biodiesel is contained in chromatographic
data which need to be obtained in other requirements. SANS 1935 seems to
acknowledge this, because Annex A allows that the iodine value can be calculated
from chromatographic results (see Section \ref{sec:ChromDetUnsat}).

Because determining iodine value is a mature technology and therefore relatively
simple, it is tempting to think of it as a test that might be useful to small
biodiesel producers. Unfortunately the iodine value is determined by the
feedstock, so that for a feedstock from a certain vegetable oil crop there is
unlikely to be any significant variation in iodine value, no matter the
production process. Iodine value might be useful as a simple indicator of
feedstock quality when the biodiesel is produced from waste vegetable oil, which
might contain a variety of oils from different origins.

The prescribed method for the requirement is SANS 54111 (or, equivalently, EN
14111). In this method a known excess of Wijs's reagent (iodine chloride in
acetic acid) is added to a weighed sample. The reaction mixture is then treated
with potassium iodide, which converts the excess ICl to I\textsubscript{2},
which is then titrated with potassium thiosulfate. The titration is followed
potentiometrically and the equivalence point determined from the titration
curve.

\section{SANS 1935: Chemical properties: Electrochemical determination}

\subsection{Water content.}

The compounds that comprise biodiesel are much more polar than those of
petrodiesel. Therefore, much more water can dissolve in biodiesel than can
dissolve in petrodiesel. This water has several deleterious effects on the
quality of biodiesel. It encourages the growth of micro-organisms, allows
hydrolysis, and increases corrosivity

The level of water specified in the requirements of SANS 1935 is lower than the
solubility of water, and therefore refers to dissolved water. Free, visible
water is excluded by paragraph 4.1.4.

The method specified by SANS 1935 for the determination of water in biodiesel is
ISO 12937. This standard prescribes the well-known Karl Fischer titration used
for the determination of water in solvents. This is a \keyword{coulometric}
titration, which means that electricity is used as titrant. The amount of charge
(current integrated over time)



\section{SANS 1935: Chemical properties: Spectroscopic determination}

\subsection{Group I metals}

The most common catalysts in biodiesel production are sodium or potassium
hydroxides (NaOH and KOH) or alkoxides (CH\textsubscript{3}ONa and
CH\textsubscript{3}OK). These catalysts are polar and will dissolve in the
glycerol byproduct of biodiesel production. But some may remain in the biodiesel
itself, and needs to be removed. This cleanup can be done by washing with water,
adsorbent columns, or selective membranes.

SANS 54108 is the method specified for the determination of sodium, and SANS
54109 the method specified for potassium. Both are flame atomic absorption
spectroscopy methods. Alternatively, EN 14538 may be used to determine sodium
and potassium simultaneously with calcium and magnesium (see
Section\label{sec:GroupIIMetals}).

\subsection{Group II metals}
\label{sec:GroupIIMetals}

Fatty acids neutralized by alkali and alkaline earth metal hydroxides form
\keyword{soaps}. If these metals are present in the feedstock, as catalyst, or
in washing water they can form soaps with the fatty acids in the biodiesel.
These soaps can form deposits in engines that can affect operations. For
example, deposits of calcium soaps have been reported to cause injectors to
stick \autocite{Pischinger2000}.

SANS 1935 prescribes EN 14538 as the method for determining Group II metals.
This is an optical emission spectroscopy method: a sample of the biodiesel is
diluted in kerosene, and injected into a inductively coupled argon plasma. The
emissions at certain wavelengths are compared to the emissions of solvent
containing known concentrations of the metals.

\subsection{Phosphorus}

The low levels of phosphorus expected to be present in biodiesel should not
affect a diesel engine's performance, but it can have a detrimental effect on
the exhaust treatment system by forming ash that can clog filters and reactive
species that can reduce catalyst effectiveness.
 
SANS 54107 is the prescribed method for determining phosphorus in biodiesel. A
sample of biodiesel is dissolved in xylene, and the solution introduced in
aerosol form into an inductively coupled argon plasma. The high temperature of
the plasma causes phosphorus atoms and/or ions to emit radiation. This emission
is measured at a certain wavelength, and compared to emissions from solutions
with known concentrations.

\subsection{Sulfur}

The amount of sulfur in biodiesel is limited not because it affects the fuel's
performance, but because the fuel must be compatible with emission control
systems and not be more polluting in terms of sulfur emissions than petrodiesel.
Most biodiesel feedstocks are naturally low in sulfur and are therefore unlikely
to exceed the limits.

There are two prescribed tests for sulfur. ISO 20846 is a UV fluorescence
method, while ISO 20884 is an X-ray fluorescence methods. The quantum-mechanical
mechanism is the same for both methods: a chemical species absorbs energy from a
photon which puts it in an activated state. The species then returns to a state
of lower energy, emitting a photon of different energy.
In the case of ISO 20846 the species is gaseous SO\textsubscript{2} (obtained by
combusting the sample), and the activating photons are from the ultraviolet part
of the electromagnetic spectrum. In the case of ISO 20884 the chemical species
are the bound form of the sulfur as found in the biodiesel, and the activating
photons are from the X-ray region of the electromagnetic spectrum.

\section{SANS 1935: Chemical properties: Chromatographic determination}
\label{sec:ChromDet}

\subsection{Methanol content}

Methanol in biodiesel increases its flash point, and it is an indicator of poor
production process control.

SANS 1935 requires a maximum mass fraction fo \SI{0.2}{\percent}. The prescribed
test method is contained in SANS 54110, and involves heating a sealed vial
partly filled with biodiesel to \SI{80}{\celsius}. A portion of the headspace
vapour is taken and injected into a gas chromatograph. The amount of methanol is
quantified by comparing the methanol peak to either an internal or external
standard.

\subsection{Ester content and Linolenic acid methyl ester content}

As prescribed in Paragraph 3 of SANS 1935, biodiesel must consist of fatty acid
methyl esters. The first line of Table 1 quantifies this requirement as a
minimum of \SI{96.5}{\percent} mass fraction. At the same time, there is a
requirement that methyl linoleate do not exceed \SI{12}{\percent}, because this
compounds detracts greatly from oxidative stability. Both compounds can be
determined by the same chromatographic method.

The specified method is SANS 54103. This document refers to ISO 5508:1990, which
has been withdrawn and superceded by ISO 12966-4:2015. ISO 5508 and ISO 12966
describe gas chromatographic methods for the determination of FAMEs. ISO 5508 is
obsolete: it gives conditions for packed GC columns and thermal conductivity
detectors, two technologies which are now rarely found in the chromatography
lab. Both methods, however, requires polar stationary phases. ISO 5508 

\subsection{Glyceride content}
\label{sec:Glycerides}

Glycerol (propane-1,2,3-triol) is the ``backbone'' of the oil molecules that
constitute the feedstock for biodiesel production. Each hydroxyl group can form
an ester bond with a fatty acid, and when all three have a fatty acid bound to
it is a `fat molecule'. IUPAC recommends that such a molecule is called a
tri-O-acylglycerol\autocite[Nic2009], but by long-established custom they are
called triglycerides. The conversion of a triglyceride to FAMEs is a stepwise
process, with one ester bond at a time being methanolized. This means that
during the reaction process, there will also be di- and monoglycerides (di- and
mono-O-acylglycerols) in the reaction mixture. If the reaction is not well
controlled then these glycerides will be found in the final product.

The presence of glycerides is of course an indicator of an incomplete
transesterification reaction, but it has further negative effects. In
particular, during cold weather, or in petroleum-blended biodiesel, some
dissolved impurities might precipitate, in particular the monoglycerides
\autocite{Dunn2009,Plata2015}. This precipitate might block filters or otherwise
interfere with engine performance.

SANS 1935 specifies that the mono-, di- and triglyceride content of biodiesel
must determined with a procedure compliant with SANS 54105 (or, equivalently, EN
14105). In this method, the biodiesel sample is treated with MSTFA
(2,2,2-Trifluoro-N-methyl-N-(trimethylsilyl)acetamide)


MSTFA figure

\chemfig{-[:138]N-[:84]=^[:156]N-[:228]=[:300](-[:240](-[:180]N(-[:240]%
)-[:120](-[:60]N(-[:120])-)=[:180]O)=[:300]O)-[:12]\phantom{N}}

MSTFA is a \keyword{derivatization reagent}: It reacts with the hydroxyl
hydrogen atoms in the glycerides to dimethylsilyl derivatives. The molecule is
then much more inert and will not interact with the column and the stationary
phase, yielding peaks with better shapes. 

\begin{figure}
\centering
\includegraphics[width=\textwidth]{Figures/MSTFA.png}
\decoRule

\caption[MSTFA]{N-methyl-N-(trimethylsilyl)-trifluoroacetamide (MSTFA)}

\label{fig:MSTFA}
\end{figure}

The trifluoroacetamide group is a good leaving group, and upon nucleophilic
attack by the hydroxyl oxygen its bond with the trimethylsilyl group breaks,
leaving the TMS bound to the oxygen. The labile hydrogen atom is now replaced by
the TMS group, which is inert and will not interact with polar entities in the
stationary phase or column, which will lead to improved peak shapes. 

\subsection{Free Glycerol}

Free glycerol is one of the products of the transesterification of plant oils to
produce biodiesel. It is a polar compound, which naturally separates from
non-polar biodiesel, and any excess remaining dissolved in the biodiesel is
removed during the washing step. Free glycerol contributes to injector coking. 

Inappropriate processing may leave excess free glycerol in the biodiesel, and therefore

Free glycerol can be determined by the same chromatographic procedure prescribed
for the determination of the other glycerides in SANS 54105 (see Section
\ref{sec:Glycerides}, but SANS 1935 also offers the option of SANS 54016. This
standard uses a liquid-liquid extraction of biodiesel with a mixture of ethanol,
water, and hexane. The free glycerol transfers quantitatively to the bottom
layer, which is then analyzed with a gas chromatographic method.


\subsection{Polyunsaturated methyl esters}
\label{sec:ChromDetUnsat}

\section{Referring documents}

\subsection{Petroleum Products Act}

SANS 1935 is referred to in the Petroleum Products act, which makes it illegal. 

\section{Fast Gas Chromatography}

\todo{autocite Blumberg 1997}

\todos